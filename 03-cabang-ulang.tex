\chapter{Struktur Percabangan dan Perulangan}

\section{Tujuan}

Pada Bab ini diharapkan mahasiswa memahami konsep Percabangan dan Perulangan serta implementasinya pada bahasa pemrograman Java.

\section{Pengantar}

Kondisi percabangan adalah kondisi dimana alur dari logika program memiliki dua atau lebih pilihan yang harus dijalankan, msaing-masing pilihan akan mengakibatkan hasil yang berbeda. Istilah lain yang biasa disebut untuk mengungkapkan ini adalah seleksi.

Sedangkan kondisi perulangan adalah kondisi dimana suatu alur program perlu melakukan beberapa pekerjaan yang berulang untuk beberapa siklus tertentu.

Implementasi untuk kedua konsep tersebut mampu dilakukan dalam bahasa pemrograman apapun, namun kita akan mencoba mengimplementasikannya di bahasa pemrograman Java.

\section{Praktek}

\subsection{Percabangan}

Percabangan di Java dapat dideklarasikan melalui beberapa cara, mari kita bahas macamnya satu satu.

\subsubsection{Operator \textit{Ternary}}

Apabila kita memiliki kasus cabang yang sederhana, yang hasilnya dapat langsung dikembalikan dan disimpan dalam sebuah variabel, kita dapat menggunakan operator \textit{ternary}. Format yang digunakan untuk dekalrasi operator \textit{ternary} ini adalah seperti berikut :

\begin{lstlisting}
(a) ? b : c
\end{lstlisting}

Keterangan dari kode tersebut adalah seperti berikut :

\begin{tabular}{| l | l |}
\hline
Huruf & Keterangan \\
\hline
a & seleksi yang hasilnya dapat bernilai \texttt{true} atau \texttt{false} \\
\hline
b & nilai yang dikembalikan apabila pernyataan pada huruf \textbf{a} \\
& bernilai \texttt{true} atau benar \\
\hline
c & nilai yang dikembalikan apabila pernyataan pada huruf \textbf{a} \\
& bernilai \texttt{false} atau salah \\
\hline
\end{tabular}

Perhatikan contoh kode berikut :

\begin{lstlisting}
public class Aplikasi {
  public static void main(String args[]) {
    int a = 10;
    int b = 13;
    
    String result = (a % 2 == 0) ? "bilangan genap" : "bilangan ganjil";
    System.out.println(a + " adalah " + result);

    result = (b % 2 == 0) ? "bilangan genap" : "bilangan ganjil";
    System.out.println(b + " adalah " + result);
  }
}
\end{lstlisting}

Perhatikan pada baris ke-6 dan ke-9, pada baris ini kita menggunakan operator \textit{ternary} untuk melakukan seleksi sederhana apakah sebuah bilangan seperti \texttt{10} atau \texttt{13} pada variabel \texttt{a} dan \texttt{b} merupakan bilangan genap atau bilangan ganjil.

Dengan operator \textit{ternary} kita tidak perlu menggunakan perintah \texttt{if} yang begitu panjang, cukup deklarasikan dalam satu baris kode, dengan cara mencari hasil sisa bagi dengan 2 (dua), apabila nilainya adalah 0 (nol), maka akan mengembalikan teks "bilangan genap", namun bila hasilnya tidak 0 (nol) maka akan mengembalikan teks "bilangan ganjil".

\subsubsection{Blok Perintah \texttt{if}}

Blok perintah \texttt{if} ini akan melakukan percabangan atau melakukan perintah yang berada dalam bloknya apabila pernyataan yang diberikan bernilai \texttt{true} atau benar.

Contoh kodenya adalah seperti berikut :

\begin{lstlisting}
if(a) b;
\end{lstlisting}

Bila pernyataan pada bagian \texttt{a} bernilai \texttt{true}, maka pernyataan pada bagian \texttt{b} akan dijalankan, namun bila bernilai \texttt{false} maka pernyataan pada bagian \texttt{b} akan dilewati.

Bentuk lain dari perintah \texttt{if} apabila kita membutuhkan lebih banyak baris kode yang dieksekusi pada bagian \texttt{b} apabila pernyataan pada bagian \texttt{a} bernilai \texttt{true} adalah seperti berikut :

\begin{lstlisting}
if(a) {
  b;
  c;
}
\end{lstlisting}

Sehingga apabila pernyataan pada bagian \texttt{a} bernilai \texttt{true}, maka pernyataan di dalam kurung kurawal (yaitu pada bagian \texttt{b} dan \texttt{c}) dapat dijalankan.

Contoh implementasi kodenya adalah seperti berikut :

\begin{lstlisting}
public class Aplikasi {
  public static void main(String args[]) {
    int a = 10;
    if(a % 2 == 0) {
      System.out.println(a + " adalah bilangan genap");
      System.out.println("ini masih dari dalam struktur if");
    }
    System.out.println("akhir aplikasi");
  }
}
\end{lstlisting}

Perhatikan bahwa kode pada baris ke-5 dan ke-6 akan tercetak bila program dijalankan, namun bila nilai pada variabel \texttt{a} kita ubah menjadi bilangan ganjil, maka kedua baris tersebut akan dilewati, karena pernyataan pada parameter \texttt{if} tidak mengembalikan nilai \texttt{true}.

\subsubsection{Blok Perintah \texttt{if...else}}

Dengan menggunakan perintah \texttt{if}, maka apabila parameter yang diberikan bernilai \texttt{false}, maka aplikasi akan melewati begitu saja. Bagaimana bila nilai pada parameter \texttt{if} bernilai \texttt{false} namun kita tetap akan menangani hasilnya? 

Solusi dari permasalahan tersebut ada pada blok perintah berikut :

\begin{lstlisting}
if(a) b;
else c;
\end{lstlisting}

Seperti pada bahasan sebelumnya, bahwa bila bagian \texttt{a} bernilai \texttt{true} maka pernyataan pada bagian \texttt{b} akan dijalankan, namun bila hasil pada bagian \texttt{a} bernilai \texttt{false}, maka yang dijalankan adalah pernyataan pada bagian \texttt{c}.

Format lain apabila kita memerlukan lebih dari 1 (satu) baris perintah pada bagian \texttt{b} dan \texttt{c}, cukup berikan kurung kurawal untuk memberikan tanda blok yang dikerjakan, formatnya menjadi seperti berikut :

\begin{lstlisting}
if(a) {
  b;
  c;
} else {
  d;
  e;
}
\end{lstlisting}

Namun pada contoh tersebut, apabila nilai pada bagian \texttt{a} bernilai \texttt{true} maka yang dikerjakan adalah pernyataan pada blok pertama, yaitu bagian \texttt{b} dan \texttt{c}, namun bila bernilai \texttt{false} maka yang dikerjakan adalah bagian \texttt{d} dan \texttt{e}.

Contoh implementasinya adalah seperti berikut :

\begin{lstlisting}
public class Aplikasi {
  public static void main(String args[]) {
    int a = 13;
    
    if(a % 2 == 0) {
      System.out.println(a + " adalah bilangan genap");
    } else {
      System.out.println(a + " adalah bilangan ganjil");
    }
    System.out.println("akhir aplikasi");
  }
}
\end{lstlisting}

Tentu saja hasil dari baris program di atas adalah tercetaknya \texttt{a} dengan keterangan \texttt{adalah bilangan ganjil}.

Selain bentuk sederhana seperti itu, perintah \texttt{if} dan \texttt{if...else...} pun sebetulnya bisa dibuat bertingkat, sehingga dapat melakukan percabangan atau seleksi beberapa kondisi dalam satu alur.

\subsubsection{Blok perintah \texttt{switch...case}}

Perintah \texttt{switch...case} ini digunakan apabila kita memiliki beberapa alternatif pilihan selain dalam bentuk \texttt{true} dan \texttt{false}. 

Struktur perintah ini adalah seperti berikut :

\begin{lstlisting}
switch(a) {
  case b:
    c;
    break;
  case d:
    e;
    break;
  default:
    f;
}
\end{lstlisting}

Dari kode di atas, yang akan dilakukan seleksi hasil adalah pada bagian \texttt{a}, apabila hasil pemeriksaan seleksi pada bagian \texttt{a} menghasilkan nilai \texttt{b}, maka yang akan dieksekusi adalah bagian \texttt{c}, sedangkan apabila hasil seleksi \texttt{a} merupakan nilai \texttt{d}, maka yang akan dijalankan adalah pada bagian \texttt{e}, terakhir apabila tidak ada satu nilai pun yang cocok, maka akan dijalankan blok kode yang berada pada bagian \texttt{f}.

Pilihan blok baris \texttt{default} sebetulnya adalah pilihan, boleh disertakan, atau tidak disertakan pun tidak apa-apa. Kemudian pemberian perintah \texttt{break} pada tiap akhir \texttt{case} adalah karena apabila sebuah blok \texttt{case} dijalankan, maka setelah baris akhir dari \texttt{case} tersebut tidak ada \texttt{break}, maka akan dilanjutkan ke \texttt{case} berikutnya.

Contoh implementasinya adalah seperti kode berikut ini :

\begin{lstlisting}
public class Aplikasi {
  public static void main(String args[]) {
    int pilihan = 2;
    
    switch(pilihan) {
      case 1:
        System.out.println("Anda memilih angka 1");
        break;
      case 2:
        System.out.println("Anda memilih angka 2");
        break;
      case 3:
        System.out.println("Anda memilih angka 3");
        break;
      default:
        System.out.println("Tidak ada pilihan");
    }
  }
}
\end{lstlisting}

Dari contoh diatas, isi variabel \texttt{pilihan} sudah kita tentukan terlebih dahulu, yaitu \texttt{2}, sehingga hasil keluarannya dapat kita tebak, yaitu menjalankan perintah pada baris ke-10.

Cobalah ganti isi variabel \texttt{pilihan} dengan angka lain, kemudian jalankan programnya.

\subsection{Perulangan}

Struktur perulangan pun dalam bahasa pemrograman Java memiliki beberapa macam bentuk, mari kita bahas apa saja bentuknya.

\subsubsection{Blok perintah \texttt{for}}

Struktur \texttt{for} ini membutuhkan 3 (tiga) parameter, formatnya adalah seperti berikut :

\begin{lstlisting}
for(a; b; c) {
  d;
}
\end{lstlisting}

Pada bagian \texttt{a} akan berisi inisialisasi nilai yang akan dilakukan iterasi atau perulangan, pada bagian \texttt{b} merupakan pemeriksaan logika apakah iterasi akan dilanjutkan atau tidak, bila bernilai \texttt{true} maka akan dilanjutkan, bila \texttt{false} maka iterasi akan dihentikan.

Pada bagian \texttt{c} akan berisi \textit{counter} yang akan dikerjakan di tiap akhir siklus masing-masing iterasi, sedangkan pada bagian \texttt{d} adalah kondisi atau pernyataan yang akan dijalankan di tiap siklus iterasi.

Contoh implementasi kodenya adalah seperti berikut :

\begin{lstlisting}
public class Aplikasi {
  public static void main(String args[]) {
    for(int i=1; i<=5; i++) {
      System.out.println("data ke-" + i);
    }
  }
}
\end{lstlisting}

Pada kode di atas, kita melakukan inisiasi variabel \texttt{i} yang bertipe data \textit{integer} dengan nilai \texttt{1}, kemudian prosesnya akan melakukan pemeriksaan logika dengan pernyataan \texttt{i<=5}, bila hasilnya bernilai \texttt{true} maka proses berlanjut dengan menjalankan blok perintah yang ada di dalam kurung kurawal, bila \texttt{false} maka perintah yang berada di dalam kurung kurawal akan dilewati.

Setiap 1 (satu) siklus pengerjaan iterasi, maka perintah \texttt{i++} akan dijalankan diakhir siklus, kemudian kembali lagi ke pemeriksaan logika apakah hasilnya masih bernilai \texttt{true} atau \texttt{false}.

\subsubsection{Blok perintah \texttt{do...while}}

Blok perintah ini akan menjalankan minimal satu siklus iterasi, karena pemeriksaan logika untuk meneruskan atau menyudahi proses iterasi berikutnya berada di akhir siklus iterasi. Formatnya adalah seperti berikut :

\begin{lstlisting}
do {
  a;
} while(b);
\end{lstlisting}

Blok baris \texttt{a} adalah yang dikerjakan dalam sebuah siklus iterasi, dan pada bagian \texttt{b} adalah pemeriksa logika yang apabila bernilai \texttt{true}, maka siklus iterasi berikutnya dikerjakan, namun bila isinya bernilai \texttt{false} maka iterasi selesai.

Contoh implementasi di Java untuk jenis iterasi ini adalah seperti berikut :

\begin{lstlisting}
public class Aplikasi {
  public static void main(String args[]) {
    int i = 1;
    do {
      System.out.println(i++);
    } while(i < 6);
  }
}
\end{lstlisting}

Kode di atas, pada baris ke-3 akan melakukan inisiasi nilai pada variabel \texttt{i} dengan angka \texttt{1}, kemudian siklus awal iterasi akan dikerjakan seperti pada baris ke-5, yaitu mencetak isi dari variabel \texttt{i}, kemudian ada tanda \texttt{++} yang artinya setelah perintah pada baris ini dikerjakan, variabel \texttt{i} akan dijumlahkan dengan 1 (\textit{increment}), setelah itu akan melakukan siklus iterasi berikutnya sampai nilai pada variabel \texttt{i} bernilai \texttt{6} yang artinya sudah tidak memenuhi persamaan pada baris ke-6, dengan kata lain bernilai \texttt{false}.

\subsubsection{Blok perintah \texttt{while...}}

Sama seperti bentuk iterasi sebelumnya, yaitu \texttt{do...while}, namun kali ini pemeriksaan logika yang menentukan apakah iterasi dikerjakan atau tidak ada di awal siklus iterasi. Bentuk blok perintahnya adalah seperti berikut :

\begin{lstlisting}
while(a) {
  b;
}
\end{lstlisting}

Pada bentuk di atas, pada bagian \texttt{a} akan diperiksa terlebih dahulu hasil operasi logikanya, bila bernilai \texttt{true}, maka perintah yang ada pada bagian \texttt{b} akan dikerjakan, namun bila bernilai \texttt{false} iterasi akan dilewatkan.

Contoh implementasi kodenya adalah seperti berikut :

\begin{lstlisting}
public class Aplikasi {
  public static void main(String args[]) {
    int angka;
    Scanner sc = new Scanner(System.in);

    System.out.print("Masukkan angka : ");
    angka = sc.nextInt();
    int i=0;
    while(i < angka) {
      System.out.println("datanya : " + i++);
    }
  }
}
\end{lstlisting}

Kali ini kita memanfaatkan kelas \texttt{Scanner} untuk menerima masukkan dari pengguna, pada baris ke-4, kita siapkan instan dari kelas \texttt{Scanner} dengan sumber data dari input pengguna, dalam hal ini papan ketik.

Pada baris ke-6, kita memberikan informasi ke pengguna untuk memasukkan sebuah angka, yang kemudian pada baris ke-7 kita simpan angka yang telah dimasukkan oleh pengguna ke variabel \texttt{angka}.

Pada baris ke-8, kita buat variabel \texttt{i} dan kita isikan nilai awalnya adalah \texttt{0} (nol), pada blok baris ke-9 sampai ke-11, kita mulai melakukan pencetakan isi dari variabel \texttt{i} sampai nilainya sama dengan \texttt{angka} yang telah dimasukkan oleh pengguna.

\section{Kesimpulan}

Bahwa percabangan digunakan apabila kita ingin melakukan seleksi terhadap sebuah nilai, yang kemudian menentukan alur logika aplikasi yang dikerjakan. Sedangkan perulangan dapat kita gunakan apabila kita membutuhkan sebuah kode yang dijalankan berulang untuk beberapa siklus tertentu.

\section{Tugas}

Buatlah sebuah aplikasi sederhana, yang terdiri dari 3 (tiga) menu, judul menunya adalah seperti berikut :

\begin{enumerate}
  \item Tambah
  \item Kurang
  \item Keluar
\end{enumerate}

Menu ini akan terus berulang, sampai pengguna memilih atau memasukkan angka 3 (tiga). 

Bila pengguna memilih angka 1 (satu) maka variabel yang telah disiapkan akan ditambahkan dengan 1 (satu) kemudian ditampilkan di layar, lalu menampilkan menu ini kembali.

Bila pengguna memilih angka 2 (dua), variabel yang telah disiapkan akan dikurangi dengan 1 (satu) kemudian ditampilkan di layar dan kembali memunculkan menu tersebut.