\chapter{Paket}

\section{Tujuan}

Pada Bab ini diharapkan mahasiswa memahami konsep dari paket (\textit{package}) dan mampu mengimplementasikan konsep tersebut pada bahasa pemrograman Java.

\section{Pengantar}

Paket pada paradigma pemrograman berorientasi objek digunakan untuk mengelompokkan beberapa kelas yang mirip atau sejenis, bisa dianalogikan bahwa ini adalah sebuah kandar dengan nama berkas yang sejenis di dalamnya.

Fungsi paket yang lain adalah agar tidak terjadi deklarasi ambigu dari sebuah kelas, misalnya, bila kita ingin mendeklarasikan 2 (dua) atau lebih kelas dengan nama yang sama namun dengan tujuan atau fungsi yang berbeda, maka cukup menggunakan penamaan paket untuk membedakan bahwa kedua kelas tersebut memang berbeda secara fungsi.

Yang perlu dicatat adalah bahwa penamaan paket di Java akan mengikuti penamaan struktur kandar di \textit{file system}, jadi nama paket akan mengikuti nama kandar-nya.

\section{Praktek}

Kali ini akan kita coba implementasikan pengguna paket ini dan bagaimana cara memanfaatkan penamaan paket ini pada kelas \texttt{Aplikasi}. 

Pertama kita perlu membuat kandar / \textit{folder} dengan nama \texttt{data}. Nama kandar ini tentu saja harus sama dengan nama paket yang akan kita gunakan, karena Java mengikuti penamaan struktur kandar di \textit{file system} yang kita gunakan.

Di dalam kandar \texttt{data}, kita membuat sebuah kelas dengan nama \texttt{Mahasiswa}, isi deklarasi kelasnya adalah seperti berikut :

\begin{lstlisting}
package data;

public class Mahasiswa {
  private String nama;
  private String nim;
  
  public Mahasiswa(String nim, String nama) {
    this.nim = nim;
    this.nama = nama;
  }
  
  public void cetak() {
    System.out.println(nim + " : " + nama);
  }
}
\end{lstlisting}

perhatikan deklarasi paket pada baris ke-1, penamaan paket ini mengikuti penamaan pada kandar yang telah kita buat sebelumnya, perhatikan besar kecilnya huruf karena ini berpengaruh.

Kelas tersebut hanya mendefinisikan 2 (dua) properti atau atribut \texttt{nim} dan \texttt{nama}, kemudian pada konstruktornya langsung ditetapkan 2 (dua) parameter untuk mengisi atribut itu, terakhir kita berikan \textit{method} \texttt{cetak} untuk melakukan pencetakan isi dari atribut \texttt{nim} dan \texttt{nama} ke layar.

Selanjutnya, di luar kandar \texttt{data}, sejajar dengan kandar ini kita buat kelas \texttt{Aplikasi}. Isi dari berkas \texttt{Aplikasi.java} ini adalah seperti berikut :

\begin{lstlisting}
import data.Mahasiswa;

public class Aplikasi {
  public static void main(String args[]) {
    Mahasiswa[] mhs = {
      new Mahasiswa("19001", "tamami"),
      new Mahasiswa("19002", "diva"),
      new Mahasiswa("19003", "nabila")
    };

    for(Mahasiswa mahasiswa : mhs) {
      mahasiswa.cetak();
    }
  }
}
\end{lstlisting}

Perhatikan baris pertamanya yang menggunakan perintah \texttt{import} untuk menyertakan kelas \texttt{Mahasiswa} pada kelas \texttt{Aplikasi}, apabila kelas \texttt{Mahasiswa} berada dalam satu kandar yang sama dengan kelas \texttt{Aplikasi}, penggunaan perintah \texttt{import} ini tidak perlu, berhubung letak kelas \texttt{Mahasiswa} berada dalam paket (kandar) \texttt{data}, maka kita perlu mendeklarasikan dengan perintah \texttt{import}.

\textit{Compile} dan jalankan kode di atas sehingga hasil yang kita dapat seharusnya akan menampilkan 3 (tiga) data mahasiswa yang telah kita definisikan pada larik \texttt{mhs}.

\section{Kesimpulan}

Bahwa penggunaan paket dibutuhkan agar tidak ada definisi kelas yang konflik pada saat implementasi. 

Deklarasi penamaan paket di bahasa pemrograman Java akan mengikuti struktur pada \textit{file system}, sehingga perlu diperhatikan penamaan kandar pada \textit{file system} yang digunakan.

\section{Tugas}

Dari tugas pada Bab 2, simpan kelas \texttt{Anggota} dalam paket \texttt{data}, kemudian tunjukkan cara memanggilnya, serta tunjukkan pula hasilnya.