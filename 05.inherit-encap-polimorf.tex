\chapter{Inheritance, Encapsulation, dan Polimorphism}

\section{Tujuan}

Pada Bab ini diharapkan mahasiswa memahami konsep \textit{inheritance} (pewarisan), \textit{Encapsulation}, dan \textit{Polimorphism}, serta bagaimana implementasi ketiga konsep tersebut pada bahasa pemrograman Java.

\section{Pengantar}

Konsep \textit{inheritance} ini sering muncul dalam pembahasan sebuah paradigma pemrograman berorientasi objek, yaitu bagaimana sebuah kelas akan mewarisi atribut dan \textit{method} milik kelas di atasnya, yang kemudian hanya tinggal menambahkan perilaku unik yang lebih detail daripada kelas yang mewarisi.

Dengan kata lain, kelas yang mewarisi, akan memiliki seluruh atribut dan \textit{method} yang dideklarasikan pada kelas di atasnya.

Konsep \textit{Encapsulation} adalah aturan pada paradigma pemrograman berorientasi objek bahwa seluruh informasi detail dalam kelas perlu disembunyikan, satu-satunya cara untuk melakukan akses terhadap informasi ini dilakukan melalui \textit{interface} yang kita kenal dengan istilah \textit{method} atau fungsi, atau prosedur.

Polimorfisme sendiri memiliki beberapa pengertian, yang pertama adalah \textit{method} atau konstruktor yang memiliki banyak bentuk, sehingga mampu memproses objek berdasarkan tipe datanya, dalam pengertian lain yaitu sebuah \textit{method} dengan implementasi yang bermacam-macam. 

Bentuk implementasi dari polimorfisme ini sendiri nantinya dapat berupa \textit{overloading} yang telah kita bahas sebelumnya, atau \textit{overriding}.

Kita perjelas saja ketiga konsep tersebut di atas pada bagian praktek.

\section{Praktek}

\subsection{Inheritance}

Implementasi untuk \textit{inheritance} atau pewarisan ini, misalkan kita memiliki sebuah kelas dengan nama \texttt{Personal} yang nantinya sebagai pewaris terhadap kelas \texttt{Mahasiswa}. 

Deklarasi untuk kelas \texttt{Personal} ini kita simpan dalam paket \texttt{entitas}, berikut adalah isi kode dari kelas \texttt{Personal} :

\begin{lstlisting}
package entitas;

public class Personal {

  private String nik;
  private String nama;
  
  public Personal() {
    nik = "3376000";
    nama = "tidak ada";
  }
  
  public Personal(String nik, String nama) {
    this.nik = nik;
    this.nama = nama;
  }
  
  public void cetak() {
    System.out.println(nik + " : " + nama);
  }
  
  // getter and setter
  // .... 

}
\end{lstlisting}

Kelas \texttt{Personal} ini memiliki 2 (dua) properti atau atribut dengan nama \texttt{nik} dan \texttt{nama}, memiliki 2 (dua) konstruktor, dan 5 (lima) buah \textit{method}, yang 1 (satu) terlihat (yaitu \textit{method} \texttt{cetak}), dan yang lain adalah \textit{method} aksesor yang sengaja tidak disertakan karena terlalu makan banyak tempat.

Selanjutnya kita deklarasikan kelas \texttt{Mahasiswa} pada paket yang sama, yaitu paket \texttt{entitas}. Isi atau deklarasi kelasnya adalah seperti berikut :

\begin{lstlisting}
package entitas;

public class Mahasiswa extends Personal {
  
  private String nim;
  
  public Mahasiswa(String nim) {
    super();
    this.nim = nim;
  }
  
  public Mahasiswa(String nim, String nik, String nama) {
    super(nik, nama);
    this.nim = nim;
  }
  
}
\end{lstlisting}

Pada saat kita deklarasikan kelas \texttt{Mahasiswa} pada baris ke-3, kita melihat ada perintah baru, yaitu \texttt{extends}, perintah inilah yang digunakan untuk menunjukkan bahwa kelas \texttt{Mahasiswa} akan mewarisi atribut dan \textit{method} milik kelas \texttt{Personal}.

Kemudian pada konstruktor \texttt{Mahasiswa} kita melihat ada perintah \texttt{super()} yang artinya sebetulnya adalah memanggil konstruktor tanpa parameter dari kelas \texttt{Personal}, yang tentunya secara otomatis seluruh deklarasi atribut dan \textit{method} akan ditempelkan pada kelas \texttt{Mahasiswa} ini.

Sekarang kita perhatikan kondisi kelas \texttt{Aplikasi} yang nantinya akan membentuk objek dari kelas \texttt{Mahasiswa} ini, kelas \texttt{Aplikasi} akan dideklarasikan di luar paket \texttt{entitas}, deklarasinya adalah seperti berikut :

\begin{lstlisting}
import entitas.Mahasiswa;

public class Aplikasi {
  public static void main(String args[]) {
    Mahasiswa[] mhs = {
      new Mahasiswa("3376001", "19001", "tamami"),
      new Mahasiswa("3376002", "19002", "diva"),
      new Mahasiswa("3376003", "19003", "nabila")
    };

    for(Mahasiswa mahasiswa : mhs) {
      mahasiswa.cetak();
    }
  }
}
\end{lstlisting}

Perhatikan bahwa pada baris ke-12, ada pemanggilan \textit{method} \texttt{cetak}, padahal pada deklarasi milik kelas \texttt{Mahasiswa}, \textit{method} tersebut sama sekali tidak ada. Hal ini karena kelas \texttt{Mahasiswa} sebetulnya mewarisi seluruh atribut dan \textit{method} dari kelas \texttt{Personal}.

\subsection{Encapsulation}

\textit{Encapsulation} atau enkapsulasi sendiri sebetulnya sudah kita implementasikan di bagian sebelumnya, kita pratinjau kembali kode dari kelas \texttt{Personal} berikut :

\begin{lstlisting}
package entitas;

public class Personal {
  
  private String nik;
  private String nama;
  
  public Personal() {
    nik = "3376000";
    nama = "tidak ada";
  }  
  
  public Personal(String nik, String nama) {
    this.nik = nik;
    this.nama = nama;
  }
  
  public void cetak() {
    System.out.println(nik + " : " + nama);
  }
  
  // getter and setter
  // ...
  
}
\end{lstlisting}

Pada kelas tersebut, kita telah menyembunyikan atribut \texttt{nik} dan \texttt{nama}, dan memberikan \textit{method} untuk melakukan akses terhadap kedua properti atau atribut tersebut dengan \textit{method getter} dan \textit{setter}.

Sampai disini kita telah mengimplementasikan konsep enkapsulasi dari paradigma pemrograman berorientasi objek dengan cara menyembunyikan detail dari kelas, dan hanya membuka \textit{interface} yang dibutuhkan oleh kelas lain. Kedepan kita akan menggunakan pola ini sebagai standar.

\subsection{Polimorphism}

Seperti dijelaskan sebelumnya bahwa pada konsep \textit{polimorphism} atau polimorfisme, konstruktor atau sebuah \textit{method} mampu melakukan operasi yang berbeda-beda sesuai dengan tipe data objek yang akan di proses, atau sebuah \textit{method} yang memiliki banyak implementasi berbeda.

Untuk definisi pertama sudah kita bahas pada Bab sebelumnya bahwa sebuah konstruktor, yang sebetulnya adalah \textit{method} juga, dapat memiliki berbagai macam bentuk dengan jumlah parameter sebagai pembeda. Sedangkan untuk pengertian kedua, kita coba implementasikan seperti berikut.

Kita akan coba perluas implementasi kelas \texttt{Personal}, selain kelas \texttt{Mahasiswa}, kita akan membuat kelas \texttt{Dosen} yang merupakan pewaris dari kelas \texttt{Personal} juga, berikut isi dari kelas \texttt{Dosen} :

\begin{lstlisting}
package entitas;

public class Dosen extends Personal {

    private String nidn;

    public Dosen(String nidn, String nama) {
        super();
        this.nidn = nidn;
        setNama(nama);
    }

    public void cetak() {
        System.out.println(nidn + " : " + getNama());
    }

}
\end{lstlisting}

Lalu kita ubah deklarasi kelas \texttt{Mahasiswa} menjadi seperti berikut :

\begin{lstlisting}
package entitas;

public class Mahasiswa extends Personal {

    private String nim;

    public Mahasiswa(String nim, String nama) {
        super();
        this.nim = nim;
        setNama(nama);
    }

    public void cetak() {
        System.out.println(nim + " : " + getNama());
    }

}
\end{lstlisting}

Perhatikan bahwa masing-masing kelas, baik kelas \texttt{Mahasiswa} dan kelas \texttt{Dosen} memiliki \texttt{method} \texttt{cetak()}, sekarang kita lihat bagaimana kode pada kelas \texttt{Aplikasi} yang telah kita modifikasi seperti berikut :

\begin{lstlisting}
import entitas.*;

public class Aplikasi {
  public static void main(String args[]) {
    Mahasiswa[] mhs = {
      new Mahasiswa("19001", "tamami"),
      new Mahasiswa("19002", "diva"),
      new Mahasiswa("19003", "nabila")
    };

    Dosen[] dosen = {
      new Dosen("1984001", "Dosen A"),
      new Dosen("1991001", "Dosen B"),
      new Dosen("1989002", "Dosen C")
    };

    System.out.println("Daftar Mahasiswa:");
    for(Personal personal : mhs) {
      personal.cetak();
    }

    System.out.println("\nDaftar Dosen:");
    for(Personal personal : dosen) {
      personal.cetak();
    }
  }
}
\end{lstlisting}

Pada baris ke-1, karena kita membutuhkan kelas \texttt{Mahasiswa} dan kelas \texttt{Dosen}, maka kita perlu melakukan \texttt{import}, kita menggunakan tanda \texttt{*} (bintang) disini hanya untuk meringkas bahwa seluruh kelas yang berada dalam paket \texttt{entitas} akan kita gunakan pada kelas ini, bentuk lain dengan tujuan yang sama dapat dideklarasikan seperti berikut :

\begin{lstlisting}
import entitas.Mahasiswa;
import entitas.Dosen;
import entitas.Personal;
\end{lstlisting}

Pada blok baris ke-5 sampai ke-9, kita membuat sebuah larik dengan isi 3 (tiga) objek dari kelas \texttt{Mahasiswa}, kemudian pada blok baris ke-11 sampai ke-15, kita pun membuat sebuah larik dengan isi 3 (tiga) objek dari kelas \texttt{Dosen}.

Pada blok baris ke-18 sampai baris ke-20, dan baris ke-23 sampai ke-25, kita menggunakan kelas \texttt{Personal} untuk menampilkan datanya, namun objek yang diambil berbeda, yang satu dari kelas \texttt{Mahasiswa}, dan yang lain dari kelas \texttt{Dosen}. 

Karena masing-masing kelas memiliki implementasi yang berbeda, maka hasil keluaran yang kita terima pun menjadi berbeda, inilah yang disebut polimorfisme, berikut hasil keluaran yang seharusnya tampil pada saat kita menjalankannya :

\begin{lstlisting}
Daftar Mahasiswa:
19001 : tamami
19002 : diva
19003 : nabila

Daftar Dosen:
1984001 : Dosen A
1991001 : Dosen B
1989002 : Dosen C
\end{lstlisting}

\section{Kesimpulan}

Bahwa konsep pewarisan disediakan dalam paradigma pemrograman berorientasi objek untuk mempermudah pemrograman dengan cara menggunakan kelas yang sudah ada untuk kemudian dikembangkan kembali, jadi tidak perlu membangunnya dari awal.

Enkapsulasi ada pada paradigma pemrograman berorientasi objek untuk menjaga keamanan data, fleksibilitas, dan memudahkan pemeliharaan, jadi jangan sampai data pada properti terubah secara sembarangan tanpa terfilter melalui \textit{interface} yang disediakan oleh si pembuat kelas dalam bentuk \textit{method}.

Sedangkan adanya polimorfisme memungkinkan pemrogram mendefinisikan banyak konstruktor dan \textit{method} untuk digunakan oleh pemrogram lain yang ingin menggunakan, di sisi lain, membuka peluang pemrogram lain untuk melakukan implementasi yang berbeda sesuai dengan kebutuhan.

\section{Tugas}

Pada sebuah tempat pelayanan penitipan hewan, akan dibuat sebuah aplikasi yang melakukan manajemen terhadap hewan yang dititipkan.

Tugas kali ini hanya membuat sebagian kecil dari bagian besar sebuah aplikasi tersebut, berikut yang perlu dikerjakan.

Buatlah sebuah kelas \texttt{Hewan}, kelas ini memiliki sebuah konstruktor dengan 2 (dua) buah parameter yang berisi nomor identitas dan nama pemilik dari \texttt{Hewan} tersebut, dan 2 (dua) buah \textit{method}, yang pertama mengembalikan nilai berupa informasi dalam bentuk teks (String) dengan format \texttt{\{id\} : \{pemilik\}}, dan yang kedua akan mengembalikan nilai \texttt{true} atau \texttt{false} yang memberikan tanda bahwa hewan tersebut sedang dititipkan atau tidak, kita beri nama \textit{method} ini sebagai \texttt{status()}.

Kemudian buat 2 (dua) kelas, \texttt{Anjing} dan \texttt{Ikan} yang mewarisi kelas \texttt{Hewan}, kelas \texttt{Anjing} akan memiliki atribut \texttt{statusSuntikRabies}, sedangkan kelas \texttt{Ikan} akan memiliki atribut \texttt{statusGantiAir}, masing-masing kelas tersebut akan merubah isi dari \textit{method} \texttt{status} untuk menampilkan informasi apakah hewan tersebut dititipkan atau tidak, dan menampilkan informasi apakah sudah pernah disuntik rabies untuk \texttt{Anjing} dan apakah sudah pernah ganti air untuk \texttt{Ikan}.