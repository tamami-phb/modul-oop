\chapter{Inheritance, Encapsulation, dan Polimorphism}

\section{Tujuan}

Pada Bab ini diharapkan mahasiswa memahami konsep \textit{inheritance} (pewarisan), \textit{Encapsulation}, dan \textit{Polimorphism}, serta bagaimana implementasi ketiga konsep tersebut pada bahasa pemrograman Java.

\section{Pengantar}

Konsep \textit{inheritance} ini sering muncul dalam pembahasan sebuah paradigma pemrograman berorientasi objek, yaitu bagaimana sebuah kelas akan mewarisi atribut dan \textit{method} milik kelas di atasnya, yang kemudian hanya tinggal menambahkan perilaku unik yang lebih detail daripada kelas yang mewarisi.

Dengan kata lain, kelas yang mewarisi, akan memiliki seluruh atribut dan \textit{method} yang dideklarasikan pada kelas di atasnya.

Konsep \textit{Encapsulation} adalah aturan pada paradigma pemrograman berorientasi objek bahwa seluruh informasi detail dalam kelas perlu disembunyikan, satu-satunya cara untuk melakukan akses terhadap informasi ini dilakukan melalui \textit{interface} yang kita kenal dengan istilah \textit{method} atau fungsi, atau prosedur.

Sedangkan pengertian dari Polimorphism adalah pada kelas yang mewarisi atribut dan \textit{method} dari kelas di atasnya, dapat memiliki implementasi mandiri yang berbeda dari kelas diatasnya, atau bahkan dari kelas lain yang mewarisi dari kelas yang sama.

Kita perjelas saja ketiga konsep tersebut di atas pada bagian praktek.

\section{Praktek}

\subsection{Inheritance}

Implementasi untuk \textit{inheritance} atau pewarisan ini, misalkan kita memiliki sebuah kelas dengan nama \texttt{Personal} yang nantinya sebagai pewaris terhadap kelas \texttt{Mahasiswa} dan \texttt{Dosen}. 

Deklarasi untuk kelas \texttt{Personal} ini kita simpan dalam paket \texttt{entitas}, berikut adalah isi kode dari kelas \texttt{Personal} :

\begin{lstlisting}
package entitas;

public class Personal {

  private String nik;
  private String nama;
  
  public Personal() {
    nik = "3376000";
    nama = "tidak ada";
  }
  
  public Personal(String nik, String nama) {
    this.nik = nik;
    this.nama = nama;
  }
  
  public void cetak() {
    System.out.println(nik + " : " + nama);
  }
  
  // getter and setter
  // .... 

}
\end{lstlisting}

Kelas \texttt{Personal} ini memiliki 2 (dua) properti atau atribut dengan nama \texttt{nik} dan \texttt{nama}, memiliki 2 (dua) konstruktor, dan 5 (lima) buah \textit{method}, yang 1 (satu) terlihat (yaitu \textit{method} \texttt{cetak}), dan yang lain adalah \textit{method} aksesor yang sengaja tidak disertakan karena terlalu makan banyak tempat.

Selanjutnya kita deklarasikan kelas \texttt{Mahasiswa} pada paket yang sama, yaitu paket \texttt{entitas}. Isi atau deklarasi kelasnya adalah seperti berikut :

\begin{lstlisting}
package entitas;

public class Mahasiswa extends Personal {
  
  private String nim;
  
  public Mahasiswa(String nim) {
    super();
    this.nim = nim;
  }
  
  public Mahasiswa(String nim, String nik, String nama) {
    super(nik, nama);
    this.nim = nim;
  }
  
}
\end{lstlisting}

Pada saat kita deklarasikan kelas \texttt{Mahasiswa} pada baris ke-3, kita melihat ada perintah baru, yaitu \texttt{extends}, perintah inilah yang digunakan untuk menunjukkan bahwa kelas \texttt{Mahasiswa} akan mewarisi atribut dan \textit{method} milik kelas \texttt{Personal}.

Kemudian pada konstruktor \texttt{Mahasiswa} kita melihat ada perintah \texttt{super()} yang artinya sebetulnya adalah memanggil konstruktor tanpa parameter dari kelas \texttt{Personal}, yang tentunya secara otomatis seluruh deklarasi atribut dan \textit{method} akan ditempelkan pada kelas \texttt{Mahasiswa} ini.

Sekarang kita perhatikan kondisi kelas \texttt{Aplikasi} yang nantinya akan membentuk objek dari kelas \texttt{Mahasiswa} ini, kelas \texttt{Aplikasi} akan dideklarasikan di luar paket \texttt{entitas}, deklarasinya adalah seperti berikut :

\begin{lstlisting}
import entitas.Mahasiswa;

public class Aplikasi {
  public static void main(String args[]) {
    Mahasiswa[] mhs = {
      new Mahasiswa("3376001", "19001", "tamami"),
      new Mahasiswa("3376002", "19002", "diva"),
      new Mahasiswa("3376003", "19003", "nabila")
    };

    for(Mahasiswa mahasiswa : mhs) {
      mahasiswa.cetak();
    }
  }
}
\end{lstlisting}

Perhatikan bahwa pada baris ke-12, ada pemanggilan \textit{method} \texttt{cetak}, padahal pada deklarasi milik kelas \texttt{Mahasiswa}, \textit{method} tersebut sama sekali tidak ada. Hal ini karena kelas \texttt{Mahasiswa} sebetulnya mewarisi seluruh atribut dan \textit{method} dari kelas \texttt{Personal}.

\subsection{Encapsulation}

\subsection{Polimorphism}

\section{Kesimpulan}

\section{Tugas}